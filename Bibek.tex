\subsection{Deployment/Methods}
Deployment is the process of using the developed system to an effective action. In this section the complete deployment of the MagicMirror is described. To get a reflection of a user like a mirror does, reflection sol firm having around 80\% of reflection was used in front of 21- inch monitor which was used as mirror’s screen. Interact between user and the system the was done using leap motion. Leap motion detects the air gesture or finger movement as input without contacting or touching the surface. The main interface for drawing graphics editor Tux Paint was used. The leap motion SDK was used to make leap motion work on windows platform as Windows 10 was used as the environment to work for Tux Paint. After all of this setup was done the prototype was ready for use. The user should stand in front of mirror in the range of leap motion and point one finger in the mirror then starts drawing.

After the whole setup, we used two different environments for testing our prototype i.e. Controlled environment and Natural environment.

In Controlled environment the testing of the prototype was done by the people or the group of people which are invited by us. The MagicMirror was deployed in the makerspace. For controlled environment, there were also two testcase. First case is that when the mirror was blank and the second one was when there was already some picture in the mirror. The main motive of controlled environment testing was to test the usability testing. There were 24 users invited as an individual and 11 users in a group of 2 or 3. All the users invited were the student of HiØ and they have to answer the questionnaire. The questionnaire is mainly to rate how enjoyable the product is and how motivated the user were to use the prototype. They need to give the answer in form rating. The rating is from 1 (not at all) to 5 (very enjoyable/motivated). We consider the scores 4 and greater as "enjoy"/"motivated".
 
Similarly, the MagicMirror was deployed in the hallway in front of the cafeteria in the University for Natural environment. The hallway was chosen for this testing as it is used almost by all students and staff of all faculties. Therefore, we have a wider range of participants. The main motive of the Natural testing was to gain the attention of the random people passing the hallway to do field studies, to observe how the people interact with the mirror without any prior information. We did not count the exact number of participants involved in this testing as we only want to observe how user interact with the MagicMirror.
