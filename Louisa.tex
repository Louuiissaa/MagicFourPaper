\section{Early Results}
The questionnaire was not able to show a difference in motivation depending on the initial state of the mirror. Therefore, our hypothesis people would be more motivated to write on the paper if the surface is blank must be rejected. Our second hypothesis that people in a group enjoy the use of the magic mirror more can be assumed. Eventhough the average amount of enjoyment is only slightly higher for people in a small group, the difference between the min and max value shows that people in a small group enjoy it more.

The observations mainly showed two points we will declare in the following. 

 Firstly, people stop to look at the mirror. During our observations, people who were not alone stopped more often than persons who were alone. We could also observe that people who walked in the direction of the cafeteria also stopped for a longer time than the person who was walking towards the classrooms. This might be due to the lack of time when people have to go to classes.

Secondly, people left quickly when the mirror did not work the expected way. We could clearly see that people had their difficulties with drawing and writing in the air with one finger. It might be that for people it is not intuitive to write something without touching the surface the person want to write on.

Besides the results from the questionnaire, it was also noticable during the observations, people who started testing the prototype with a non-blank surface, asked for a new blank surface if the previous drawing occupied more than \textasciitilde20\% of the surface. Due to the small given surface, the surface was most of the time too full after the use of one person/group.

\section{Future Work}
During the evaluation of the magic mirror it become clear that the air gesture is for a lot of people not intuitive. Instead they tried to touch the surface of the mirror in order to draw or write something. Therefore, an interesting approach would be how people would adapt the magic mirror if the surface is a reflective touch screen.

The second approach, which should be tested in future work, is the impact of an increased surface. Especially, the collaborative part could improve if the surface offers space to several drawings and texts at the same time. So people actually have the space to add something to a previous drawing. For the group use, the use of the magic mirror by multiple users at the same time is also an interesting aspect which shoulb be looked at in future work.
