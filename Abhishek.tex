
\section{Introduction}

Communication refers to the process of passing messages between peoples which may be verbal, non - verbal or visual. This is a collaborative process, which makes people more social and interactive. Ages ago, as a form of visual communication, people used to carve messages on stone pillars to communicate very well across time which is perfect for the common areas. This paper focuses on this same concept integrated with the technology using the mirror, a reflective surface.

Over the past couple of decades, growing technology has revolutionised the world. This is constantly having a massive impact on the human race at every stage of societal development. Not only computers connected objects and sensors are getting smarter but also any object around us and the way we interact with an emerging world is getting smarter and more enhanced. We picked a mirror as our interaction device and made it smarter with additional technologies. We merged the concept of passing messages with mirror so we call it ''Magic Mirror''. So far, no smart mirror has been developed that makes the mirror more appealing to general users. Also, none of the smart mirrors developed has the concept of social communication. This paper suggests a new mode of passing messages which connects people that underlie social interaction.

''Passing Messages'' is a primary goal of Magic Mirror. Sharing the message just with the reflective mirror using hand gestures without any additional devices or IoT, just makes it so simple to use. One of the purposes of this concept is making people more social not just by sharing messages but also this lets the multiple participants work together to create something which is obviously a team building.
\section{Related Work}
The proposed Magic Mirror represents a social and collaborative interaction interface that provides a platform to share the message with the personal or group contribution. And there has been number of projects based on the smart mirror which are related and driven in similar direction. Philips Homelab is a permanent fully functional home laboratory built to study how people interact with interactive and automated home environment. Smart Lighting system, augmented broadcasts, smart music collection, smart memory browsing system, interactive mirror are some of the setups in homelab. Interactive mirror can be placed in room or washroom and can be personalized according to the need of the end user. Every person can customise the content (i.e. cartoons for children's, adults can get live news feeds and updates on weathers, traffics, mail etc).

Researchers from Griffith University conducted the early exploration of the suitability of Leap Motion controller in the project entitled ''The Leap Motion controller: A view on sign language''. They conducted this research for Auslan (Australian Sign Language) and they found out the leap motion to be perfect for basic signs but not appropriate for complex signs.

Another project named FitMirror as carried out by students of Ulm University. They created a fitmirror system to help users to get up in the morning and get motivated for the day. They built the recognition system for fun and normal exercises which made the exercise fun for even middle and low motivated persons. They used Kinect for emotion recognition and hand trace for different games.

In comparison, to various projects and works mentioned above, our proposed aim is different which is to develop social interaction between people by sharing messages using the combined functionality of mirror and leap motion.