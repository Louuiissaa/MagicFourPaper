% This is samplepaper.tex, a sample chapter demonstrating the
% LLNCS macro package for Springer Computer Science proceedings;
% Version 2.20 of 2017/10/04
%
\documentclass[runningheads]{llncs}
%
\usepackage{graphicx}
% Used for displaying a sample figure. If possible, figure files should
% be included in EPS format.
%
% If you use the hyperref package, please uncomment the following line
% to display URLs in blue roman font according to Springer's eBook style:
% \renewcommand\UrlFont{\color{blue}\rmfamily}

\begin{document}
%
\title{Magic Mirror}
%
%\titlerunning{Abbreviated paper title}
% If the paper title is too long for the running head, you can set
% an abbreviated paper title here
%
\author{Abhishek Pokhrel\inst{1}\orcidID{0000-1111-2222-3333} \and
Bibek Acharya\inst{1}\orcidID{1111-2222-3333-4444} \and
Tuan Tai Le\inst{1}\orcidID{2222--3333-4444-5555} \and
Louisa Pabst\inst{2}\orcidID{182322}}
%
\authorrunning{Pokhrel, Acharya, Le, Pabst}
% First names are abbreviated in the running head.
% If there are more than two authors, 'et al.' is used.
%
\institute{H{{\o}}gskolen i {{\O}}stfold, B R A Veien 4, 1783 Halden, Norway \and
Otto-Friedrich-Universit{\"a}t, Kapuzinerstra{\ss}e 16, 96047 Bamberg, Germany
}
%
\maketitle              % typeset the header of the contribution
%
\begin{abstract}
The abstract should briefly summarize the contents of the paper in
150--250 words.

\keywords{First keyword  \and Second keyword \and Another keyword.}
\end{abstract}
%
%
%
\newpage
\section{Introduction}
\subsection{RQ/Problem Statement}
\subsection{Conceptual Framework}
\subsection{Previous Work}
\section{Methods}
\subsection{Technology}
\subsection{Evaluation}
\section{Results}
\section{Discussion}
\section{Conclusion}
\subsection{Future Work}

%
% ---- Bibliography ----
%
% BibTeX users should specify bibliography style 'splncs04'.
% References will then be sorted and formatted in the correct style.
%
% \bibliographystyle{splncs04}
% \bibliography{mybibliography}
%
\begin{thebibliography}{8}
\bibitem{ref_article1}
Author, F.: Article title. Journal \textbf{2}(5), 99--110 (2016)

\bibitem{ref_lncs1}
Author, F., Author, S.: Title of a proceedings paper. In: Editor,
F., Editor, S. (eds.) CONFERENCE 2016, LNCS, vol. 9999, pp. 1--13.
Springer, Heidelberg (2016). \doi{10.10007/1234567890}

\bibitem{ref_book1}
Author, F., Author, S., Author, T.: Book title. 2nd edn. Publisher,
Location (1999)

\bibitem{ref_proc1}
Author, A.-B.: Contribution title. In: 9th International Proceedings
on Proceedings, pp. 1--2. Publisher, Location (2010)

\bibitem{ref_url1}
LNCS Homepage, \url{http://www.springer.com/lncs}. Last accessed 4
Oct 2017
\end{thebibliography}

\end{document}
